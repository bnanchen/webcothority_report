\documentclass[11pt, a4paper, twoside, openright]{book} %draft

\usepackage{graphicx,color}
\usepackage{amssymb, amsmath, array}
\usepackage{cite}
\usepackage[T1]{fontenc}

\usepackage{listings}
\usepackage{color}

\begin{document}

\input{cover}

\tableofcontents
\newpage

\chapter{Introduction}
\section{History}
%First present the work at Dedis Cothority and Cosi
Distributed cryptography spreads the operation of a cryptosystem among a group
of servers in a fault-tolerant way~\cite{definition}.\\
The DEDIS lab at EPFL created the Cothority project, which implements decentralized
and distributed cryptographic protocols.\\

TODO:presentation of the report!!!!!!!!!!!!!!!!!!!!!!!!!!!!!!!!!

%Myself
\section{Aims and Goals}
The goal of this semester project is to furnish a web-interface to the Cothority
project.\\
The aims stated at a summer meeting with Mr.Linus Gasser are to create an
interface with status' conodes, be able to send a file for a collective signature
and to verify a signature.\\


\chapter{Analysis}
The first part of the semester project was to acquire knowledge in Javascript and
HTML and become familiar with the JQuery library. For delivering a skeleton of
the website before the beginning of the semester.\\
Next thing to settle was the communication between the website and a conode.
The Javascript Web APIs contain all the necessary to handle this problem.
The object Websocket~\cite{websocketPage} offers all the tools to create a
communication between a browser and a server.\\
The Cothority's approach for serializing structured data is a Google's Protobuf-like.
As said on the website of Google's Protobuf:''Protocol buffers are a flexible,
efficient, automated mechanism for serializing structured data \\- think XML, but
smaller, faster, and simpler.''~\cite{protobufDefi}.\\
Taking this into account and knowing that Google's Protobuf doesn't support generated
code in Javascript, the choice was made on the library protobuf.js~\cite{protobufjs}.
It is a pure Javascript implementation of Google's Protobuf. It uses the same format
of \\.proto file.

\begin{lstlisting}[caption={example of \\.proto file}]
 message Foo{
            required bytes a = 1;
            required bytes b = 2;
        }
\end{lstlisting}

%talk about the implementation of the Status Part
\section{Status part}
The Javascript Web APIs contain all the necessary to handle the communication
part. The object Websocket~\cite{websocketPage} offers all the tools to create a
communication between a browser and a server and send/receive data. All the elements send to conodes are \\.proto files.\\
First it is necessary to establish the connection. A connection is established
with each conode.\\
An empty \\.proto file is send in a Blob object containing the \\.proto file in bytes.\\

\begin{lstlisting}[caption={empty \\.proto file}]
  message Request {
  }
\end{lstlisting}

The request being sent the webpage waits for a response. The response will be the
status of the concerning conode. It is received as a \\.proto file.

\begin{lstlisting}[caption={response \\.proto file}]
  message ServerIdentity{
    				required bytes public = 1;
    				required bytes id = 2;
    				required string address = 3;
    				required string description = 4;
				}

  message Response {
    				map<string, Status> system = 1;
    				optional ServerIdentity server = 2;

				    message Status {
        				map<string, string> field = 1;
    				}
				}
\end{lstlisting}[caption{response \\.proto file}]

%talk about the problem with map in protofile and after runGenerator

% Google Chrome with async

\bibliography{report}{}
\bibliographystyle{ieeetr}

\end{document}
